\documentclass{llncs}
\usepackage[utf8]{inputenc}
\usepackage{graphicx}
\usepackage{color}
\usepackage{url}
\usepackage[spanish]{babel}
\usepackage{epstopdf}

\begin{document}

%%%%%%%%%%%%%%%%%%%%%%%%%%%%%%%   TITLE   %%%%%%%%%%%%%%%%%%%%%%%%%%%%%%%

\title{Estado del arte en predicción de series temporales}

%%%%%%%%%%%%%%%%%%%%%%%%%%%%%%%   AUTHORS   %%%%%%%%%%%%%%%%%%%%%%%%%%%%%%%

\author{J.J. Asensio, P.A. Castillo}
\authorrunning{J.J. Asensio et al.}

\institute{
Departamento de Arquitectura y Tecnología de Computadores \\
ETSIIT, CITIC-UGR \\
Universidad de Granada, España \\
\email{\{asensio, pacv\}@ugr.es}
}
% la dirección de pablo es pgarcia y tú no sé si tienes - JJ
\maketitle
%
%%%%%%%%%%%%%%%%%%%%%%%%%%%%%%%%%   ABSTRACT   %%%%%%%%%%%%%%%%%%%%%%%%%%%%%%%%%
%
\begin{abstract} 
bonito abstract
\end{abstract}


%
%%%%%%%%%%%%%%%%%%%%%%%%%%%%%%%   INTRODUCTION   %%%%%%%%%%%%%%%%%%%%%%%%%%%%%%%
%
\section{Introducción}
\label{sec:intro}


\bibliographystyle{splncs}


\bibliography{bibliography}


\end{document}

%%Está chulo el trabajo, el STATE OF ART es de premio :)
